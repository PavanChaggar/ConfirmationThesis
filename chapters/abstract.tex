\begin{center}
    \vspace*{1.5cm}
    \Large \bfseries  Abstract
\end{center}

Alzheimer's disease is a devastating neurological disorder characterised by
progressive and staged grey matter atrophy, leading to a broad dwindling of
brain functions. At present, Alzheimer's disease remains incurable, with disease
modifying interventions limited. Toxic forms of two proteins, amyloid-$\beta$
(\AB) and $\tau$-protein (\TP), are believed to underlie Alzheimer's disease by
aggregating into amyloid plaques and tau tangles, which in turn cause cellular
disruption and death. The spreading of these proteins through the brain,
particularly that of tau-protein, is highly conserved among Alzheimer's patients
and is correlated with the progression of grey-matter atrophy and symptom onset.
These features have been subject to extensive modelling work over the past
decade, with results showing that simple models of protein transport and growth
are able to reproduce clinical patterns. In recent work, we have shown that a
Bayesian modelling pipeline can be applied to calibrate models with
$\tau$-protein PET data, estimate parametric uncertainty and predict patient
trajectories. However, the simple models that have been analysed with patient
data exclude important features about \TP PET observations that may bias
simulations and result in erroneous predictions for patient trajectories. To
address this issue, we develop model of \TP propagation that incorporates
important features of \TP PET, namely observed baseline and carrying capacities.
Using this regionally specific model of \TP PET, we use a Bayesian pipeline to
calibrate a population level hierarchical model for different patient groups.
Our results highlight differences in dynamics across patient groups, showing
higher rates of transport in early stage disease and faster growth rates in late
stage disease. This work provides the first generative model of \TP PET and
accurately fits the subject data across different patient groups. With a robust
model of \TP in hand, I then outline plans for future work on incorporating
other important parts of AD pathology, such as \AB interactions and genetic risk
factors. 